\documentclass[a4paper,10pt]{article}
\usepackage[utf8]{inputenc}
\usepackage{amsmath}

\title{Programa 2:\
	   Analizador Sintáctico}

\author{\textbf{Integrantes:}\\
		Ahuacatitan García Juan Carlos \\
		Henkel Magnus}

\begin{document}

\maketitle


\section{}
Decir que elementos gramaticales componen el conjunto de símbolos no  terminales
y cuáles el de terminales.\\

\noindent
N = \{ program declaraciones sentencias sentencia tipo lista-variables\\
\-\hspace{1cm}init sentencias sentencia sentenciaIF sentenciaWhile\\
\-\hspace{1cm}sentenciaAsig condicion expresion \}\\

\noindent
E = \{int float double char bool id numero caracter true false if\\
\-\hspace{1cm}else while ; , = ( ) { } + - * / \% > < ! \}\\

\noindent
S = program

\section{}
Asignar un símbolo para representar a cada uno de los elementos no
terminales, por ejemplo: expresión será o representada por E y programa por P.
Esto con el motivo de simplificar la notación de la gramática y sea de manejo
más sencillo.
\\

\noindent
    INT \rightarrow int\\
    FLOAT \rightarrow float\\
    DOUBLE \rightarrow double\\
    CHAR \rightarrow char\\
    BOOL \rightarrow bool\\
    ID \rightarrow id\\
    NUM \rightarrow numero\\
    CAD \rightarrow caracter\\
    TRUE \rightarrow true\\
    FALSE \rightarrow false\\
    IF \rightarrow if\\
    EL \rightarrow else\\
    WHI \rightarrow while\\
    PYC \rightarrow ;\\
    CO \rightarrow ,\\
    ASIG \rightarrow =]\\
    PA \rightarrow (\\
    PC \rightarrow )\\
    LLA \rightarrow \{\\
    LLC \rightarrow \}\\
    MAS \rightarrow +\\
    MENOS \rightarrow - \\
    MUL \rightarrow *\\
    DIV \rightarrow /\\
    MOD \rightarrow \%\\
    MAQ \rightarrow >\\
    MEQ \rightarrow <\\

\section{}


\section{}
Decir que elementos gramaticales componen el conjunto de símbolos no  terminales
y cuáles el de terminales.\\

\noindent
N = \{ program declaraciones sentencias sentencia tipo lista-variables\\
\-\hspace{1cm}init sentencias sentencia sentenciaIF sentenciaWhile\\
\-\hspace{1cm}sentenciaAsig condicion expresion \}\\

\noindent
E = \{int float double char bool id numero caracter true false if\\
\-\hspace{1cm}else while ; , = ( ) { } + - * / \% > < ! \}\\

\noindent
S = program

\section{}
Asignar un símbolo para representar a cada uno de los elementos no
terminales, por ejemplo: expresión será o representada por E y programa por P.
Esto con el motivo de simplificar la notación de la gramática y sea de manejo
más sencillo.
\\

\noindent
    INT \rightarrow int\\
    FLOAT \rightarrow float\\
    DOUBLE \rightarrow double\\
    CHAR \rightarrow char\\
    BOOL \rightarrow bool\\
    ID \rightarrow id\\
    NUM \rightarrow numero\\
    CAD \rightarrow caracter\\
    TRUE \rightarrow true\\
    FALSE \rightarrow false\\
    IF \rightarrow if\\
    EL \rightarrow else\\
    WHI \rightarrow while\\
    PYC \rightarrow ;\\
    CO \rightarrow ,\\
    ASIG \rightarrow =]\\
    PA \rightarrow (\\
    PC \rightarrow )\\
    LLA \rightarrow \{\\
    LLC \rightarrow \}\\
    MAS \rightarrow +\\
    MENOS \rightarrow - \\
    MUL \rightarrow *\\
    DIV \rightarrow /\\
    MOD \rightarrow \%\\
    MAQ \rightarrow >\\
    MEQ \rightarrow <\\
\section{}
Decir que elementos gramaticales componen el conjunto de símbolos no  terminales
y cuáles el de terminales.\\

\noindent
N = \{ program declaraciones sentencias sentencia tipo lista-variables\\
\-\hspace{1cm}init sentencias sentencia sentenciaIF sentenciaWhile\\
\-\hspace{1cm}sentenciaAsig condicion expresion \}\\

\noindent
E = \{int float double char bool id numero caracter true false if\\
\-\hspace{1cm}else while ; , = ( ) { } + - * / \% > < ! \}\\

\noindent
S = program

\section{}
Asignar un símbolo para representar a cada uno de los elementos no
terminales, por ejemplo: expresión será o representada por E y programa por P.
Esto con el motivo de simplificar la notación de la gramática y sea de manejo
más sencillo.
\\

\noindent
    INT \rightarrow int\\
    FLOAT \rightarrow float\\
    DOUBLE \rightarrow double\\
    CHAR \rightarrow char\\
    BOOL \rightarrow bool\\
    ID \rightarrow id\\
    NUM \rightarrow numero\\
    CAD \rightarrow caracter\\
    TRUE \rightarrow true\\
    FALSE \rightarrow false\\
    IF \rightarrow if\\
    EL \rightarrow else\\
    WHI \rightarrow while\\
    PYC \rightarrow ;\\
    CO \rightarrow ,\\
    ASIG \rightarrow =]\\
    PA \rightarrow (\\
    PC \rightarrow )\\
    LLA \rightarrow \{\\
    LLC \rightarrow \}\\
    MAS \rightarrow +\\
    MENOS \rightarrow - \\
    MUL \rightarrow *\\
    DIV \rightarrow /\\
    MOD \rightarrow \%\\
    MAQ \rightarrow >\\
    MEQ \rightarrow <\\
\section{}
Decir que elementos gramaticales componen el conjunto de símbolos no  terminales
y cuáles el de terminales.\\

\noindent
N = \{ program declaraciones sentencias sentencia tipo lista-variables\\
\-\hspace{1cm}init sentencias sentencia sentenciaIF sentenciaWhile\\
\-\hspace{1cm}sentenciaAsig condicion expresion \}\\

\noindent
E = \{int float double char bool id numero caracter true false if\\
\-\hspace{1cm}else while ; , = ( ) { } + - * / \% > < ! \}\\

\noindent
S = program

\section{}
Asignar un símbolo para representar a cada uno de los elementos no
terminales, por ejemplo: expresión será o representada por E y programa por P.
Esto con el motivo de simplificar la notación de la gramática y sea de manejo
más sencillo.
\\

\noindent
    INT \rightarrow int\\
    FLOAT \rightarrow float\\
    DOUBLE \rightarrow double\\
    CHAR \rightarrow char\\
    BOOL \rightarrow bool\\
    ID \rightarrow id\\
    NUM \rightarrow numero\\
    CAD \rightarrow caracter\\
    TRUE \rightarrow true\\
    FALSE \rightarrow false\\
    IF \rightarrow if\\
    EL \rightarrow else\\
    WHI \rightarrow while\\
    PYC \rightarrow ;\\
    CO \rightarrow ,\\
    ASIG \rightarrow =]\\
    PA \rightarrow (\\
    PC \rightarrow )\\
    LLA \rightarrow \{\\
    LLC \rightarrow \}\\
    MAS \rightarrow +\\
    MENOS \rightarrow - \\
    MUL \rightarrow *\\
    DIV \rightarrow /\\
    MOD \rightarrow \%\\
    MAQ \rightarrow >\\
    MEQ \rightarrow <\\
\end{document}
